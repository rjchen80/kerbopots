The design for \emph{kerbopots} in the final tournament follows a rather different rationale from the previous iterations. The new design, which we call ZABot, scales up the aggressiveness of the bot in the pre-flop stage.

As in the previous iterations, ZABot makes the decision to fold, check, call or raise based on the adjusted discounted equity. ZABot raises its opponents (if allowed) when equity is relatively low, calls or checks its opponents when equity is relatively high, and uses the previous pre-flop strategy when equity is moderate. During flop, turn and river, ZABot also performs a semi-bluff if its hand is 1-draw away from straight or flush and the board cards are 1-draw or 2-draw away from straight or flush. Otherwise ZABot behaves as its predecessors. The semi-bluff of ZABot is meant to create the impression that it has a straight or flush and scare its opponents to fold.

One of the main features of ZABot is that it will almost never fold to its opponents when its estimated equity is low, and will force its opponents to either fold or call to its raises. This new strategy is designed to force its opponents to fold with relatively low cards. ZABot has an advantage when playing conservative players, but it also fails when facing opponents that resemble ZeroBot, which base does not fold when its opponents continually raise. We will discuss the advantages and disadvantages of this strategy in the Discussion section.

For the final tournament, we assume that most bots in the competition will be relatively conservative and will fold under pressure. If this is the case, ZABot will have an advantage.

For the MIT Pokerbot Competition, we developed three main classes of \emph{kerbopots}. ZeroBot, our simplest bot, makes decisions based on the equity of its hands against random other hands, and decides the amount of its bets according to the Kelly Criterion. OneBot, our second iteration bot, estimates its equity against its opponents by discounting its equities against random according to its opponents actions. ZABot, our final bot, adopts more aggressive pre-flop strategies to force its opponents to fold during pre-flop.

Interestingly, in general, OneBot beats ZeroBot, ZABot beats OneBot, and ZeroBot beats ZABot. We now briefly discuss this phenomenon. OneBot is able to beat ZeroBot because it predicts whether ZeroBot has a good hand and folds when it believes its hands are worse, thereby reducing the amount of money lost when dealt an inferior hand. When it has a better hand than ZeroBot, it will put more money into the pot, and thus often beats Zerobot. On the other hand, when OneBot plays ZABot, ZABot raises its opponents when it believes its chances of winning the hand are low, and calls or checks its opponents when it believes its chances of winning are high. This strategy is more risky and only succeeds when it forces the opponents to fold. However, the aggressiveness often works against OneBot, which believes ZABot has a good hand while in reality it doesn't. This new strategy allows ZABot to use low equity hands to fold OneBot's hands, and keep relatively high equity hands alive. In this way, ZABot would enter post-flop stages with a higher percentage of winning hands than OneBot. In general, ZABot can be expected to be effective when playing against conservative players. On the contrary, if ZABot is playing a bot which considers only the equity of his own hand and does not speculate its opponents' equities, such as ZeroBot, ZABot will not be able to fold its opponents' good hands, and will lose more often by entering post-flop with inferior hands and a large pot size.

In conclusion, the three generations of \emph{kerbopots} make up an interesting circle of dominance. This circle demonstrates how the outcome of one given strategy will depend on the opponents' strategies. Ideally, to handle different classes of opponents, an adaptive strategy should be implemented based on the opponents' behaviors.
